\subsection{Основное}\label{subsec:major}
Программа написана на языке программирования C++(так как требовалась максимальная скорость), сборка осуществляется с помощью CMake.
Проект можно скомпилировать под Windows (компилятор MSVC) и под Linux (тестировалось на g++-10).

Код хранится в \href{https://github.com/donRumata03/Painter}{guthub-репозитории (кликабельно)}.

\subsection{Библиотеки}\label{subsec:libs}

\subsubsection{OpenCV}
Для работы с изображениями используется OpenCV, но не модуль машинного обучения, а лишь примитивные операции с изображениями:
прочтение из популярных форматов, сохранение в них, хранение и копирование матрицы пикселей и т.д.

\subsubsection{Pythonic}
Для работы проекта также необходима библиотека «\href{https://github.com/donRumata03/pythonic}{pythonic}»: она написана мной, подключается также через CMake.
Она отвечает за базовые функции и структуры данных.
Я использую её во всех более или менее крупных проектах на C++.
В ней на данный момент есть:
\begin{itemize}
    \item Простые вспомогательные функции для работы со строками, контейнерами, форматированного вывода
    \item Вызов питоновской библиотеки matplotlib для построения графиков
    \item Базовые алгоритмы наподобие бинарного поиска и дерева отрезков
    \item Функционал для работы со временем, в том числе — анализатор последовательных запусков процесса
    \item Платформонезависимая работа с кодировками и файловыми системами
    \item Примитивы для вычислительной геометрии
    \item Функции для работы со статистикой
    \item Многомерный шаблонный массив с количеством измерений, изменяемом в run-time
    \item Сглаживание функций и построение примерной функции распределения в пространстве с заданной размерностью по набору sampl-ов с помощью гауссовых ворот
    \item Функционал для работы с многопоточностью, в том числе — thread pool, умеющий снимать нагрузку с ожидающих потоков с помощью std::condition\_variable.
\end{itemize}

\subsubsection{lunasvg}
Для работы с SVG используется библиотека \href{https://github.com/sammycage/lunasvg}{lunasvg}.

\textcolor{gray}{P. S. У этой библиотеки отличный автор, он изучает проекты, в которых библиотека используется, и пишет рекомендации о best practice её использования.}

\subsubsection{PowerfulGA}
Функционал по методам оптимизации реализован мной и вынесен в отдельную репозиторию: \href{https://github.com/donRumata03/PowerfulGA}{click}
(там не только Генетический алгоритм, как можно было подумать из названия, но и симуляция отжига, градиентный спуск, метод Ньютона;
планируется добавить много других алгоритмов)
Более подробное описание в секции $\longrightarrow$ \ref{sec:opimization_algorithms}