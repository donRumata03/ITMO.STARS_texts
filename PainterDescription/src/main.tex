%! Author = Vova
%! Date = 13.07.2021

% Preamble
\documentclass[11pt]{article}


% Packages
\usepackage{amsmath}
\usepackage{hyperref}
\usepackage{polyglossia}


% Language and Font settings

% Kurale
% New Standard Old

\setdefaultlanguage{russian}
\setmainfont[Ligatures=TeX]{New Standard Old}
\newfontfamily\cyrillicfont{New Standard Old}[Script=Cyrillic]

\title{Описание программной части робота-художника}
\author{Латыпов Владимир Витальевич}
\date{\today}

% Document
\begin{document}
    \maketitle



    \section{Формулировка задачи}

    Для того, чтобы робот-художник нарисовал что-либо, ему нужно предоставить данные в определённом формате, а именно — не набор пикселей,
    как требуется для показа на мониторе, а набор «мазков»: это связано с конструкцией самого робота.
    Мазки решено было представлять в виде кривых безье второго порядка (то есть квадратичных), к которым добавлены параметры «толщина» и «цвет».

    Но на вход подаются рисунки не в векторном, а в растровом формате.
    Найти такую комбинацию мазков, которая бы лучше всего соответствовала картине/изображению — задача нетривиальная, имеющая множество решений.

    Поэтому решено было использовать эвристические алгоритмы оптимизации:
    \href{https://en.wikipedia.org/wiki/Simulated_annealing}{Генетический алгоритм} и \href{https://en.wikipedia.org/wiki/Simulated_annealing}{Симуляция отжига}.

    Функцию ошибки необходимо задать таким образом, чтобы она отражала качество полученной комбинации мазков,
    причём в любой точке направление её уменьшения соответствовало направлению улучшения результата.
    Помимо напрашивающегося \href{https://en.wikipedia.org/wiki/Mean_squared_error}{MSE}, используемого

    \begin{equation}\label{eq:equation}
        MSE = \sum_{y = 0}^{y < height} { \sum_{x = 0}^{x < width} { \sum_{c \in  \left\{ r, g, b \right\} } { \left( {\overrightarrow {rendered_{x, y}}}_c - {\overrightarrow{original_{x, y}}}_c\right)^2 }}}
    \end{equation}


    \section{Дальнейшее развитие}
    Несмотря на то, что программа уже работоспособна, есть ещё много идей и планов по её усовершенствованию:
    \begin{itemize}
        \item Внедрить \textit{быстрый пересчёт функции ошибки} — это улучшение давно напрашивается,
                но оно несколько теряет в эффективности из-за того, что в одной мутации в среднем изменяется не так мало мазков. В настоящий момент ведётся работа над этим.
        \item
    \end{itemize}

\end{document}